\documentclass[a4paper,10pt]{article}
\usepackage[utf8]{inputenc}
\usepackage{graphicx,subfigure,amsmath,xcolor,comment,url,fullpage}
\usepackage{fullpage}


\newcommand{\todo}[1]{{\color{red}(TODO: #1)}}
\newcommand{\degrees}{^\circ}
\newcommand{\mm}{\,\text{mm}}

\title{Cardiac Mechanics Benchmark: Problem Definitions}
\author{Sander Land}

\begin{document}

\maketitle

\appendix

\noindent This document is a work in progress, but feel free to forward it to collaborators.
For an up-to-date version, questions about joining the cardiac mechanics benchmark please contact \url{sander.land@kcl.ac.uk}.

\section{General}

All problems use the constitutive law by Guccione et al. \cite{guccione_finite_1995}, with strain energy function:
\begin{align}
W   =& \frac{C}{2} (e^Q - 1)  \\
Q   =&     b_f E_{11}^2 + b_t (E_{22}^2 + E_{33}^2 + E_{23}^2+ E_{32}^2) + b_{fs} (E_{12}^2 + E_{21}^2 + E_{13}^2 + E_{31}^2)
\end{align}

Where $\mathbf{E}$ is the Green strain in a local coordinate system with fibers in the $e_1$ direction.\\

In all problems the material is fully incompressible, i.e. the Jacobian $J=1$.
However, no restrictions are implied with respect to the methods used to satisfy incompressibility. %, Lagrange multipliers, penalty functions.
Likewise, participants are free to use any computational method, or type of elements in case of finite element methods.\\

Please note that in all problems the direction of the pressure boundary condition scales with the deformed normal, and its magnitude with deformed area.\\

The three problems are described below in increasing order of complexity. Solving all three is encouraged, but not required for participation in the benchmark.

\section{Problem 1 : Deformation of a bar}

\textbf{Geometry:} 
 The undeformed geometry is the region $x \in [0,10]\mm$, $y \in [0,1]\mm$, $z \in [0,1]\mm$.\\
%\textbf{Constitutive parameters:} transversely isotropic as in (Guccione et al. 1995) :  $C = 0.876~\text{kPa}, b_f  = 18.48, b_t  = 3.58, b_{fs} = 1.627$ \\
\textbf{Constitutive parameters:} Transversely isotropic, $C=2~\text{kPa}, b_f=8, b_t=2, b_{fs}=4$.\\
\textbf{Fiber direction:} Constant along the long axis, i.e. $(1,0,0)$.\\
\textbf{Dirichlet boundary conditions:}
 The left face ($x=0$) is fixed in all directions.\\
\textbf{Pressure boundary conditions:}
 A pressure of $0.004$ kPa is applied to the bottom face ($z=0$).
 
 
\section{Problem 2 : Inflation of a ventricle}

\textbf{Geometry:} 
 The undeformed geometry is defined using the parametrization for a truncated ellipsoid:\\
\begin{equation}
\mathbf{x} = 
 \left( \begin{array}{c} x\\y\\z \end{array}\right) =
  \left(\begin{array}{c}
   r_s \sin u \cos v    \\ 
   r_s \sin u \sin v  \\  
   r_l \cos u   
   \end{array}\right) 
\label{eqn:ellipsoid}
\end{equation}

The undeformed geometry is defined by the volume between:
\begin{itemize}
 \item The \emph{endocardial surface} $r_s=7 \mm, r_l=17\mm, u \in [-\pi, -\arccos \frac{5}{17}], v\in[-\pi,\pi]$, 
 \item The \emph{epicardial surface} $r_s=10 \mm, r_l=20\mm, u \in [-\pi, -\arccos \frac{5}{20}], v\in[-\pi,\pi]$
 \item The \emph{base plane} $z=5\mm$ which is implicitly defined by the ranges for $u$.
\end{itemize}
\textbf{Constitutive parameters:} Isotropic, $C=10 \text{ kPa}, b_f=b_t=b_{fs}=1$.\\
\textbf{Dirichlet boundary conditions:}
 The base plane($z=5\mm$) is fixed in all directions.\\
\textbf{Pressure boundary conditions:}
 A pressure of 10 kPa is applied to the endocardium.

\section{Problem 3 : Active contraction of a ventricle}

\textbf{Geometry, Dirichlet boundary conditions:} 
  Identical to problem 2.\\
\textbf{Fiber definition:} 
  Fiber angles range from $-90\degrees$ at the epicardial surface to $+90\degrees$ at the endocardial surface.
  They are defined using the direction of the derivatives of the parametrization of the ellipsoid in Equation \ref{eqn:ellipsoid}:
\begin{equation}
 f(u,v) =
   n\!\left( \frac{ \text{d}\mathbf{x} }{ \text{d}u } \right) \sin \alpha + 
   n\!\left(\frac{ \text{d}\mathbf{x} }{ \text{d}v } \right)\cos \alpha = 
\text{ where } n(\mathbf{v}) = \mathbf{v}/\|\mathbf{v}\|
\end{equation}

\begin{align}
\frac{ \text{d}\mathbf{x} }{ \text{d}u } =& \left(\begin{array}{c} r_s \cos u \cos v \\ r_s \cos u \sin v \\  -r_l \sin u \end{array}\right)  \\
\frac{ \text{d}\mathbf{x} }{ \text{d}v } =& \left(\begin{array}{c} -r_s \sin u \sin v \\ r_s \sin u \cos v \\ 0  \end{array}\right)  \\
r_s    =&  7 + 3t\\
r_l    =& 17 + 3t\\
\alpha =& 90 - 180t
\end{align}

Where $r_s$, $r_l$ and $\alpha$ are derived from the transmural distance $t \in [0,1]$ which varies linearly from $0$ on the endocardium and $1$ on the epicardium.
The apex ($u=-\pi$) has a fiber singularity which should not affect the converged solution. Participants are free to handle this in whatever way they choose.
MATLAB scripts implementing this parametrization to generate a regular hexahedral mesh with cubic or linear elements are available in the shared repository.\\
\textbf{Constitutive parameters:} Transversely isotropic, $C=2 \text{kPa}, b_f=8, b_t=2, b_{fs}=4$.\\
\textbf{Active contraction:} 
  The active stress is given by a constant, homogeneous, second Piola-Kirchhoff stress in the fiber direction of 60 kPa, i.e.
\[
 T = \text{passive stress} + T_a f f^T
\]
Where $T_a=60$ kPa and $f$ is the unit column vector in the fiber direction described above.\\
\textbf{Pressure boundary conditions:}
 A constant pressure of 15 kPa is applied to the endocardium.

As this is a quasi-static problem, participants are free to add active stress first, add pressure first, or increment both simultaneously in finding a solution.
 
\begin{thebibliography}{99}

\bibitem{guccione_finite_1995}
  JM Guccione, KD Costa, AD McCulloch.
  \emph{Finite element stress analysis of left ventricular mechanics in the beating dog heart}.
  Journal of Biomechanics 28(10), 1167--1177, 1995

  
\end{thebibliography}


\end{document}

